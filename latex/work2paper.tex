\documentclass[draft,grl]{/home/gdj/文档/template/agu_template/AGUTeX}
\usepackage{lineno}
\linenumbers*[1]
%  To add line numbers to lines with equations:
%  \begin{linenomath*}
%  \begin{equation}
%  \end{equation}
%  \end{linenomath*}
\usepackage{graphicx}
%uncomment the following if you want to see figures.
%\setkeys{Gin}{draft=false}
\authorrunninghead{GUO ET AL.}
\titlerunninghead{SHORT TITLE}
\authoraddr{Corresponding author: Jiuhou Lei, Department of Space Physics, 
University of Science and Technology of China, Hefei, Anhui, China. (leijh@ustc.edu.cn)}
\begin{document}

\title{The UT variation of pole thermospheric densities and its dependence on solar wind sector polarity and solar activity}
\authors{Dongjie Guo\altaffilmark{1,2},
Jiuhou Lei\altaffilmark{1,2,3}, and
Xiankang Dou\altaffilmark{1,2,3}}

\altaffiltext{1}{CAS Key Laboratory of Geospace Environment, School of Earth and Space Sciences, 
University of Science and Technology of China, Hefei, China}
\altaffiltext{2}{Mengcheng National Geophysical Observatory, University of Science and Technology of China, 
Hefei, China}
\altaffiltext{3}{Collaborative Innovation Center of Astronautical Science and Technology, Harbin, China}

\begin{abstract} 
    This study conducts a superposed epoch analysis with the thermospheric mass densities near the geographic poles 
    during the years 2002-2010 derived from the accelerometer onboard the GRACE satellite. The geographic poles are 
    preferred for multiple reasons. Firstly, the GRACE satellite passes near the poles in its every orbit. Secondly,
    the longitude and local time effects can be neglected at the poles, so we can focus on universal time (UT)
    variations. More importantly, the south pole passes near the cusp location at about 1600 UT every day. Our results
    show that the thermospheric densities near the south pole are enhanced by up to 16\% of the daily mean at about 1600 
    UT, which may result from the periodic pass of the cusp zone. It is also found that the UT variation depends on solar
    wind sector structure and solar activity. Near the south pole, it is likely to observe a more apparent and 
    enhanced UT variation when the earth is embedded in an away (-Bx, +By) solar wind sector polarity. The opposite
    applies to the north pole. In addition, stronger solar activities may enhance the UT change in thermospheric 
    densities near both the south and north poles.
\end{abstract}

\begin{article}

\section{Introduction}
    The universal time (UT)/longitudinal variations in the thermospheric density or temperature have been long 
    discussed \citep[e.g.,][]{Hedin1972, Hedin1979, Hedin1985, Xu2013a, Xu2013}. At high latitudes, the UT/longitudinal
    variations of the thermosphere are interpreted to the localized polar heating associated with soft 
    electron precipitation \citep{Hedin1972}. \citet{Xu2013} analyzed the longitudinal structure of the thermospheric
    density by using the daily mean thermospheric mass densities derived from the CHAMP and GRACE satellites. Their 
    results show that there are density peaks near the magnetic poles. It is generally difficult to distinguish the UT and 
    longitudinal variations since UT and longitude change synchronously at a fixed local time (LT). However, near the 
    geographic poles, the longitude and LT effects are not an issue. The UT effect could be isolated at this region. In the
    present work, we will focus on the pole regions.

    The geographical pole zones are alsoThe magnetic latitudes (MLAT) of the north and south geographical poles in the apex coordinates are about $83^{\circ}$ 
    and $-74^{\circ}$, respectively. The south pole is quite close to the dayside cusp region in the magnetic latitude 
    direction. The magnetic local time (MLT) of the geographical poles changes as the Earth rotates. The south pole 
    gets close to 1200 MLT at about 1600 UT. In other words, the south pole periodically passes near the dayside cusp
    at about 1600 UT. Therefore it is expected that the high altitude thermospheric density near the south pole increases at
    about 1600 UT due to enhanced energy precipitation from the magnetosphere \citep[e.g.,][]{Luhr2004, Zhang2012, 
    Deng2013}. 

\section{Conclusions} 
    In the present study, Near both north and south geographic poles, there was a clear UT variation in the
    thermospheric densities. At the south pole, the density could be enhanced by up to 16\% of the daily mean
    at about 16:00UT (~12:24 MLT). The UT variation of the pole densities might result from the periodic pass 
    of the cusp zone.

\bibliography{/home/gdj/work/bib/literature}
\bibliographystyle{/home/gdj/文档/template/agu_template/BibTeX/agufull08}
\end{article}
%------------------------------
\begin{figure}
    \centering
    \noindent\includegraphics[width=0.8\textwidth]{/data/output/fig/work2/fig2.eps}
    \caption{Superposed epoch results of $Bx$, $By$ and $Bz$ components of IMF in GSM coordinate system and the geomagnetic activity index $ap$ as a function of month and epoch time during 2002-2010. The epoch time centers at 0000 UT on the reversing day when the solar wind sector polarity changes from away from to toward (Away-Toward) or from toward to away from (Toward-Away) the Sun. }
    \label{figure1}
\end{figure}
\begin{figure}
    \centering
    \noindent\includegraphics[width=0.8\textwidth]{/data/output/fig/work2/fig25.eps}
    \caption{Relative changes in thermospheric densities near the geographic north (N) and south (S) poles ($\left|latitude\right| > 88.5^\circ$) versus epoch time. The thermospheric density data during 2002-2010 are sorted in month and solar wind sector polarity. The blue lines represent the median values, and the gray dashed lines represent the upper and lower quartiles.}
    \label{figure2}
\end{figure}
\begin{figure}
    \centering
    \noindent\includegraphics[width=0.8\textwidth]{/data/output/fig/work2/fig26.eps}
    \caption{ Magnetic local time changes versus universal time for the north and south geographical poles, respectively. The blue vertical line in the right panel at 1600 UT represents the universal time when the density near the south pole maximizes as shown in Figure 2.}
    \label{figure3}
\end{figure}
\begin{figure}
    \centering
    \noindent\includegraphics[width=0.8\textwidth]{/data/output/fig/work2/fig32.eps}
    \caption{The same as Figure 2, but the time intervals are constrained to years during 2002-2004 and 2008-2010 for the upper and lower panels, respectively}
    \label{figure4}
\end{figure}
%------------------------------
%...as shown by \textit{Kilby} [2008].
%...as shown by {\textit  {Lewin}} [1976], {\textit  {Carson}} [1986], {\textit  {Bartholdy and Billi}} [2002], and {\textit  {Rinaldi}} [2003].
%...has been shown [\textit{Kilby et al.}, 2008].
%...has been shown [{\textit  {Lewin}}, 1976; {\textit  {Carson}}, 1986; {\textit  {Bartholdy and Billi}}, 2002; {\textit  {Rinaldi}}, 2003].
%...has been shown [e.g., {\textit  {Lewin}}, 1976; {\textit  {Carson}}, 1986; {\textit  {Bartholdy and Billi}}, 2002; {\textit  {Rinaldi}}, 2003].

%...as shown by \citet{jskilby}.
%...as shown by \citet{lewin76}, \citet{carson86}, \citet{bartoldy02}, and \citet{rinaldi03}.
%...has been shown \citep{jskilbye}.
%...has been shown \citep{lewin76,carson86,bartoldy02,rinaldi03}.
%...has been shown \citep [e.g.,][]{lewin76,carson86,bartoldy02,rinaldi03}.
% Please use ONLY \citet and \citep for reference citations.
%------------------------------
\end{document}
