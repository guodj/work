\documentclass[12pt,a4paper]{article}

\usepackage{geometry}
\geometry{left=2cm, right=2cm,top=1.5cm, bottom=2.5cm}
\linespread{2}

\usepackage{graphicx}
\usepackage[auth-sc]{authblk}
\begin{document}
%--------------------------------------------------------
% title
%--------------------------------------------------------
\title{\textbf{Diurnal variation of thermospheric neutral density associated with solar wind sector structure}}
\author[1,2]{Dongjie Guo}
\author[1,2,3]{Jiuhou Lei}
\author[1,2,3]{Xiankang Dou}

\affil[1]{CAS Key Laboratory of Geospace Environment, School of Earth and Space Sciences,
University of Science and Technology of China, Hefei, China}

\affil[2]{Mengcheng National Geophysical Observatory, University of Science and
Technology of China, Hefei, China}

\affil[3]{Collaborative Innovation Center of Astronautial Science and Technology, Harbin,
China}
\date{}
\maketitle
\newpage
%---------------------------------------------------------------------------
% abstract
%---------------------------------------------------------------------------
\begin{abstract}

\end{abstract}
\newpage
%---------------------------------------------------------------------------
% conclusions
%---------------------------------------------------------------------------
\section{Conclusions}
The main conclusions of the present work can be summaried as follows:
\begin{enumerate}
    \item The thermospheric neutral densities at high latitudes derived from the CHAMP satellite showed a strong UT variation around equinox days, especially when the Earth was embedded in a solar wind sector with negative IMF $B_z$. The possible cause may be the diurnal variation of IMF $B_z$ associated with the angle between the geomagnetic and rotational axes.
    \item During the period of solar wind sector reversing in equinox days, the maximum change of thermospheric mass densities at 400 km was about $46\%$ relative to the 11 day averaged densities at  middle and low  latitudes, and it  can even reach about $66\%$ at high latitudes. Meanwhile, the globally and daily averaged thermospheric density variation was about $23\%$ [\textsl{Guo et al.}, 2015]. The density change was higher in the present work, which may be due to the more precise time and latitude resolutions of the data set.
    \item During March equinox days, the UT variation of high-latitude thermospheric neutral density was more obvious in the northern hemisphere than the southern hemisphere. Oppositely, during September equinox days, the thermospheric density showed a more obvious UT variation in the southern hemisphere than the northern hemisphere. This hemispheric asymmetry may be related to the different combined effects of IMF $B_y$ and $B_z$ in different seasons.
    \item The UT variation of the high latitude thermospheric neutral density was stronger at the southern high latitudes around September than that at the northern high latitudes around March. The northern thermospheric density in March only had a minimum value at $\sim$22:00UT; whereas the southern thermospheric density in September had a maximum value at $\sim$16:00UT and a minimum value at $\sim$06:00UT, respectively.
\end{enumerate}

\newpage
\begin{figure}[!ht]
    \begin{center}
        \includegraphics[width=0.95\textwidth]{figures/fig1}
        \caption{variations of (a) X, Y, and Z components of Interplanetary Magnetic Field (in unit of nT) in GSM coordinates, and (b) the neutral density (in unit of $10^{-12}$ kg/m$^3$) derived from the CHAMP satellite versus UT and latitude, during the period from April 15 to April 25 of the year 2003. The solar wind sector polarity reversed from away from the Sun to toward the Sun around April 20.} 
        \label{fig1}
    \end{center}
\end{figure}
\newpage

\begin{figure}[!ht]
    \begin{center}
        \includegraphics[width=0.95\textwidth]{figures/fig2}
        \caption{Same as Figure 1 but during the period from March 7 to March 17 of the year 2003. The solar wind sector polarity reversed from toward the Sun to away from the Sun around March 12.}
        \label{fig2}
    \end{center}
\end{figure}
\newpage

\begin{figure}[!ht]
    \begin{center}
        \includegraphics[width=0.95\textwidth]{figures/fig3}
        \caption{Same as Figure 1 but during the period from August 27 to September 6 of the year 2003. The solar wind sector polarity reversed from away from the Sun to toward the Sun around September 1.}
        \label{fig3}
    \end{center}
\end{figure}
\newpage

\begin{figure}[!ht]
    \begin{center}
        \includegraphics[width=0.95\textwidth]{figures/fig4}
        \caption{Same as Figure 1 but during the period from September 11 to September 21 of the year 2003. The solar wind sector polarity reversed from toward the Sun to away from the Sun around September 16.}
        \label{fig4}
    \end{center}
\end{figure}
\newpage

\begin{figure}[!ht]
    \begin{center}
        \includegraphics[width=0.95\textwidth]{figures/fig5}
        \caption{Superposed epoch results of the changes in (a) X, Y, and Z components of Interplanetary Magnetic Field (in unit of nT) in GSM coordinates, and (b) the relative  neutral density derived from the CHAMP satellite with respect to the 11-day averages, versus epoch day and latitude. The 0 epoch day denotes the time when the sector polarity reverses from away from the Sun to toward the Sun around March equinox.}
        \label{fig5}
    \end{center}
\end{figure}
\newpage

\begin{figure}[!ht]
    \begin{center}
        \includegraphics[width=0.95\textwidth]{figures/fig6}
        \caption{Same as Figure 5, but the 0 epoch day denotes the time when the sector polarity reverses from toward the Sun to away from the Sun around March equinox.}
        \label{fig6}
    \end{center}
\end{figure}
\newpage

\begin{figure}[!ht]
    \begin{center}
        \includegraphics[width=0.95\textwidth]{figures/fig7}
        \caption{Same as Figure 5, but the 0 epoch day denotes the time when the sector polarity reverses from away from the Sun to toward the Sun around September equinox.}
        \label{fig7}
    \end{center}
\end{figure}
\newpage

\begin{figure}[!ht]
    \begin{center}
        \includegraphics[width=0.95\textwidth]{figures/fig8}
        \caption{Same as Figure 5, but the 0 epoch day denotes the time when the sector polarity reverses from toward the Sun to away from the Sun around September equinox.}
        \label{fig8}
    \end{center}
\end{figure}
\newpage

\end{document}
