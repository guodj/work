\documentclass[12pt,a4paper]{article}
\usepackage{geometry}    %  页面控制
\geometry{left=3cm, right=3cm, top=1.0cm, bottom=3cm}
\usepackage{graphicx}  %插入图片
\DeclareGraphicsExtensions{.png,.eps}

\begin{document}

\section{Physical picture}
\begin{figure}[!ht]
    \begin{center}
        \includegraphics[width=0.8\textwidth]{/data/output/fig/work2/fig1}
        \caption{CHAMP and GRACE LT in every day. For some days, the results are wrong. From f12() in $\sim$/work/python/work2.py}
    \end{center}
    \label{fig1}
\end{figure}

\begin{figure}[!ht]
    \begin{center}
        \includegraphics[width=0.8\textwidth]{/data/output/fig/work2/fig7}
        \caption{ CHAMP MLT/MLat and LT/Lat distributions in January 1-10, 2005. Note the difference of Mlat between the Northern and Southern hemisphere. From f17() in $\sim$/work/python/work2.py}
    \end{center}
    \label{fig7}
\end{figure}

\begin{figure}[!ht]
    \begin{center}
        \includegraphics[width=0.8\textwidth]{/data/output/fig/work2/fig4}
        \caption{Test griddata and contourf, zooming in or out x values in contour won't change the results. From f11() in $\sim$/work/python/work2.py}
    \end{center}
    \label{fig4}
\end{figure}

\begin{figure}[!ht]
    \begin{center}
        \includegraphics[width=0.8\textwidth]{/data/output/fig/work2/fig6}
        \caption{Schematic of Mlat/Mlon(Mlt) in apex coordinates, and Lat/lon in geodetic coordinates. from f3() in $\sim$/work/python/sth.py}
    \end{center}
    \label{fig6}
\end{figure}
\newpage

\section{work2}
Superposed epoch analysis of the thermosphere density response to sector boundary crossings.
How to use the magnetic apex coordinate?
\begin{figure}[!ht]
    \begin{center}
        \includegraphics[width=0.8\textwidth]{/data/output/fig/work2/fig2}
        \caption{Imf and ap variations versus day of year and epoch days centered at sector boundary crossing date, from f15() in $\sim$/work/python/work2.py}
    \end{center}
    \label{fig2}
\end{figure}

\begin{figure}[!ht]
    \begin{center}
        \includegraphics[width=0.8\textwidth]{/data/output/fig/work2/fig3}
        \caption{CHAMP and GRACE density variations versus day of year and epoch days centered at sector boundary crossing date. The CIR and CME effects are not excluded. Away-toward: 520 up or down cases; toward-away: 534 up or down cases. From f16() in $\sim$/work/python/work2.py. }
    \end{center}
    \label{fig3}
\end{figure}

\begin{figure}[!ht]
    \begin{center}
        \includegraphics[width=0.8\textwidth]{/data/output/fig/work2/fig8}
        \caption{From top to bottom are for Northern high latitudes, low latitudes and Southern high latitudes. For away-toward condition, Jan-Apr data are selected; for toward-away condition, Jul-Oct data are selected. From f16() in $\sim$/work/python/work2.py.}
    \end{center}
    \label{fig8}
\end{figure}

\begin{figure}[!ht]
    \begin{center}
        \includegraphics[width=0.8\textwidth]{/data/output/fig/work2/fig9}
        \caption{Density variations during +/- 5 days of away-toward and toward-away sectors. Both Northern and Southern hemispheres are shown. From f18() in $\sim$/work/python/work2.py.}
    \end{center}
    \label{fig9}
\end{figure}
\end{document}
