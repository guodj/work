\documentclass[12pt,a4paper]{article}
\usepackage{geometry}    %  页面控制
\geometry{left=3cm, right=3cm, top=1.0cm, bottom=3cm}
\usepackage{graphicx}  %插入图片
\DeclareGraphicsExtensions{.png,.eps}

\begin{document}
\begin{figure}[!ht]
    \begin{center}
        \includegraphics[width=0.8\textwidth]{/data/output/eps/work2/fig1}
        \caption{CHAMP and GRACE LT in every day, from f12() in $\sim$/work/python/work2.py}
    \end{center}
    \label{fig1}
\end{figure}

\begin{figure}[!ht]
    \begin{center}
        \includegraphics[width=0.8\textwidth]{/data/output/eps/work2/fig4}
        \caption{test griddata and contourf, from f11() in $\sim$/work/python/work2.py}
    \end{center}
    \label{fig1}
\end{figure}

\newpage
Superposed epoch analysis of the thermosphere density response to sector boundary crossings.
How to use the magnetic apex coordinate?
\begin{figure}[!ht]
    \begin{center}
        \includegraphics[width=0.8\textwidth]{/data/output/eps/work2/fig2}
        \caption{Imf and ap variations versus day of year and epoch days centered at sector boundary crossing date, from f15() in $\sim$/work/python/work2.py}
    \end{center}
    \label{fig2}
\end{figure}

\begin{figure}[!ht]
    \begin{center}
        \includegraphics[width=0.8\textwidth]{/data/output/eps/work2/fig3}
        \caption{CHAMP and GRACE density variations versus day of year and epoch days centered at sector boundary crossing date. The CIR and CME effects are not excluded. Away-toward: 520 up or down cases; toward-away: 534 up or down cases. From f16() in $\sim$/work/python/work2.py. }
    \end{center}
    \label{fig3}
\end{figure}
\end{document}
