\documentclass[draft, grl]{/home/guod/Documents/template/agu_template/AGUTeX}
\usepackage{lineno}
\linenumbers*[1]
%  To add line numbers to lines with equations:
%  \begin{linenomath*}
%  \begin{equation}
%  \end{equation}
%  \end{linenomath*}
\usepackage{graphicx}
%uncomment the following if you want to see figures.
%\setkeys{Gin}{draft=false}
\authorrunninghead{GUO ET AL.}
\titlerunninghead{SHORT TITLE}
\authoraddr{Corresponding author: Jiuhou Lei, Department of Space Physics, 
University of Science and Technology of China, Hefei, Anhui, China. (leijh@ustc.edu.cn)}
\begin{document}

\title{UT variation of the pole thermospheric density by periodical passage of the polar cusp}
\authors{Dongjie Guo\altaffilmark{1,2},
Jiuhou Lei\altaffilmark{1,2,3}, and
Xiankang Dou\altaffilmark{1,2,3}}

\altaffiltext{1}{CAS Key Laboratory of Geospace Environment, School of Earth and Space Sciences, 
University of Science and Technology of China, Hefei, China}
\altaffiltext{2}{Mengcheng National Geophysical Observatory, University of Science and Technology of China, 
Hefei, China}
\altaffiltext{3}{Collaborative Innovation Center of Astronautical Science and Technology, Harbin, China}

\begin{abstract} 
    This study conducts a superposed epoch analysis with the thermospheric mass densities near
    the geographic poles during the years 2002-2010 derived from the accelerometer onboard the 
    GRACE satellite.
    The geographic poles are preferred for multiple reasons. 
    Firstly, the GRACE satellite passes near the poles in its every orbit.
    Secondly, the longitude and local time effects can be neglected at the poles, so we can focus on 
    universal time (UT) variations. 
    More importantly, the south pole passes near the cusp location at about 1530 UT every day.
    Our results show that the thermospheric densities near the south pole are enhanced by up to 16\% of
    the daily mean at about 1530 UT, which may result from the periodic pass of the cusp zone.
    It is also found that the UT variation depends on solar wind sector structure and solar activity.
    Near the south pole, it is likely to observe a more apparent and enhanced UT variation when the 
    earth is embedded in an away (-Bx, +By) solar wind sector polarity.
    The opposite applies to the north pole.
    In addition, stronger solar activities may enhance the UT change in thermospheric densities near 
    both the south and north poles.
\end{abstract}

\begin{article}

\section{Introduction}
    The universal time (UT)/longitudinal variations in the thermospheric density or temperature have 
    been long discussed \citep[e.g.,][]{Hedin1972, Hedin1979, Hedin1985, Xu2013a, Xu2013}. 
    At high latitudes, the UT/longitudinal variations of the thermosphere are interpreted to 
    the localized polar heating associated with soft electron precipitation \citep{Hedin1972}. 
    \citet{Xu2013} analyzed the longitudinal structure of the thermospheric density by using the daily mean 
    thermospheric mass densities derived from the CHAMP and GRACE satellites. 
    Their results show that there are density peaks near the magnetic poles. 
    It is generally difficult to distinguish the UT and longitudinal variations since UT and longitude change 
    synchronously at a fixed local time (LT). 
    However, near the geographic poles, the longitude and LT effects are not an issue. 
    The UT effect could be therefore isolated at this region. 
    In the present work, we will focus on the pole zones.

    The geographic poles are preferred also because they provide the opportunity to investigate thermospheric 
    density variations in the cusp region. 
    \citet{Luhr2004} found that the thermospheric density is enhanced by almost a factor of two 
    near the polar cusp.  
    Since then, a large quantity of work has devoted to investigating the energy 
    source of the density enhancement \citep[e.g.,][]{Deng2011, Knipp2011, Zhang2012, Deng2013, Zhang2015}.  
    However, no studies noticed the relationship between geographic poles and polar cusps. 
    The magnetic latitudes (MLAT) of the north and south geographic poles in the apex coordinates 
    are about $83^{\circ}$ and $-74^{\circ}$, respectively. 
    The south pole is quite close to the polar cusp in the magnetic latitude direction. 
    The magnetic local time (MLT) of geographic poles changes as the earth rotates. 
    The south pole gets close to 1200 MLT at about 1530 UT. In other words, the south 
    pole periodically passes near the dayside cusp at about 1530 UT. 
    Therefore it is expected that the high altitude thermospheric density near the south pole increases at
    about 1530 UT due to enhanced energy precipitation from the magnetosphere \citep{Luhr2004}.

    In this paper, we mainly used thermospheric densities near the geographic poles, which were derived from
    the accelerometer onboard the Gravity Recovery and Climate Experiment (GRACE) satellite. 
    A superposed epoch analysis was conducted to the data to present the UT variation of 
    the thermospheric density, which might be related to the pole's periodical passage through 
    the dayside cusp. 
    We chose the zero epoch time as 0000 UT of the day when the solar wind sector polarity 
    changes from away from to toward the sun or the opposite.
    The reason for the choice is that the interplanetary magnetic field (IMF) y component 
    exerts an effect on the density response to the cusp energy precipitation 
    \citep{Crowley2010, Knipp2011, Li2011}.

\section{Data and Method}
    In this study, we used the version 2.3 data for GRACE accelerometer-derived thermospheric total mass 
    densities during the years 2002-2010 \citep{Sutton2011}. 
    The twin GRACE satellites were launched into an near-circular orbit in March 2002.
    There are approximately 15 orbits in one day for GRACE satellites.
    The inclination of the satellites is about $89.5^\circ$.
    Therefore the GRACE satellite passes by the south and north geographic poles nearly 15 
    times in any specified day, which could well include the UT variations.
    In the present work, thermospheric mass densities for poleward of $88.5^\circ$ and $-88.5^\circ$ 
    were used to represent the north and south geographical poles.
    The initial altitude of the GRACE satellite is approximately 500 km, and it gradually decreases 
    to about 475 km in 2010. 
    During the same period, the altitude change for CHAMP satellite is from about 420 km to 
    approximately 320 km.
    The thermospheric density response to external forcings at a fixed altitude is associated with the 
    distance from the energy deposition altitude \citep{Lei2010a}.
    \citet{Clemmons2008} even found a small density depletion in the cusp relative to the adjacent 
    areas at about 250 km.
    The GRACE satellite is prefered to the CHAMP satellite partly due to the smaller height drop, 
    which can weaken the altitude effect.
    Note that the thermospheric densities are normalized to 400 km to remove the altitude change in a day.

    Events applied to the superposed epoch analysis in the present paper were the earth passage of the solar 
    wind sector boundary, which is defined by \citet{Svalgaard1976}.
    During the investigated period, there were 120 cases for sector polarity transitions from away from to 
    toward the sun and 122 cases for the opposite.
    Note that a small part of the cases were excluded due to data gap of the thermospheric density.
    The zero epoch time assigned to these events was 0000 UT on the sector polarity transition day.
    With these events and the assigned epoch time, the superposed epoch analysis could be 
    conducted to physical parameters, including both external forcings and the resultant pole 
    thermospheric density variations.

\section{Results and Discussion}
    The external solar forcings were firstly examined.
    Figure \ref{figure1} gives the superposed epoch results of $B_x$, $B_y$, and $B_z$ IMF 
    compnents in GSM coordinates and the associated geomagnetic auroral electrojet (AE) 
    index in different months. 
    The time range is $\pm3$ days from the 0 epoch time, i.e., 0000 UT on the sector polarity 
    transition day in each event.
    IMF $B_x$ and $B_y$ components varies in the way as expected from the defination of solar wind sectors.
    In the away sectors, the average values of $B_x$ and $B_y$ are $-2.5$ nT and $2.5$ nT, respectively.
    And they have the opposite sign but similar magnitude in the toward sectors.
    The seasonal variation in the IMF $B_z$ component has been interpreted by the 
    Russell-McPherron effect \citep{Russell1973}.
    In the equinox seasons, there is a large slant of the geomagnetic axis, and therefore a great
    projection of IMF $B_z$ is generated from the ideal spiral IMF in the solar equatorial plane.
    The Russell-McPherron effect also predicts that the largest southward component of IMF occurs near
    2200 UT around March equinox and 1000 UT around September equinox, which is also observed in our results
    in Figure \ref{figure1}.
    This UT variation in IMF $B_z$ component may also cause a corresponding response in thermosphere density.
    Whereas it may be covered by the cusp heating effect, and this will be discussed in detail later.
    
    The superposed epoch analysis is then conducted to the pole thermospheric mass densities derived from the 
    GRACE satellite and is illustrated in Figure \ref{figure2}.
    Before the statistical analysis, the density data are normalized to obtain the relative  density $\rho_r$:
    $\rho_r=100\frac{\rho-\rho_m}{\rho_m}\%$, where $\rho$ is the original measurement, 
    $\rho_m$ is the 6-day mean density during the period of any event.
    The events are categrized into groups according to season and sector polarity transition order for
    both the north and south pole densities as shown in Figure \ref{figure2}.
    Months between Febrary-April, August-October, May-July, November-January stand for March equinox, 
    September equinox, June solstice and December solstice, respectively.
    It is then expected that the IMF varies similarly for events in the same categary.
    It is found that there are approximately 30 events in each group, showing small distinction among groups.
    In Figure \ref{figure2}, the blue solid lines represent the median values of $\rho_r$ in 
    each group, and the gray dotted lines are the upper and lower quartiles. 
    The red verticle dashed lines are to show the times when the geographical poles reach 12:00 magnetic
    local time. 
    For the south and north poles, they are 1530 UT and 0530 UT, respectively.

    As shown in Figure \ref{figure2}, the UT variation of the densities is clearly observed 
    near both the northern and southern poles.
    The southern pole density generally maximizes at about 1530 UT when it passes the cusp region. 
    And the relative difference between the density peak and valley can reach up to $55\%$ of
    the 6-day mean.
    One may think that the UT variation may possibily result from the IMF $B_z$ variation caused by the
    Russell-McPherron effect \citep{Russell1973}. 
    And the IMF $B_z$ does present UT change shown in Figure \ref{figure1}. 
    However, the Russell-McPherron effect precits $B_z$ to maximize at 2200 UT around March equinox and 
    1000 UT around September equinox.
    This does not match the density pattern as observed in Figure \ref{figure2}.
    Therefore the Russell-McPherron effect can be ignored as to the pole density change.
    The invariant occurring time of southern density peaks leads to the interpretation that the density 
    is enhanced near the cusp due to increased magnetospheric energy input \citep{Luhr2004}.
    The above results remind us to take the cusp density enhancement into account when considering the
    UT variation of the thermosphere density at high latitudes.
    Note that at northern pole, the density peaks do not occur at a fixed UT. This may be due to the
    large distance of the northern pole from the cusp in the zonal direction.

    The UT variations of the thermospheric density exhibit a dependence on solar wind sector polarities, 
    as illustrated in Figure \ref{figure2}. 
    Near the southern pole, the densities generally present a more evident UT variation under 
    away ($-B_x$, $+B_y$) polarities than toward ($+B_x$, $-B_y$) ones.
    This result can be obtained when contrasting the density variations before and after zero
    epoch time of away-toward sectors in September and December seasons (Figure \ref{figure2}b), 
    and of toward-away sectors in March, September and December seasons (Figure \ref{figure2}d).
    For the northern pole, the opposite condition applies, especially for away-toward sectors.
    The solar wind sector dependence of the UT variation could be an effect of IMF $B_y$ and/or 
    IMF $B_x$, rather than IMF $B_z$. 
    Since $B_z$ varies differently with seasons according to Russell-McPherron effect as shown 
    in Figure \ref{figure1}.
    Our result may indicate that the cusp density increasing effect may be enhanced at the Southern
    (Northern) Hemisphere by positive (negative) IMF $B_y$.
    And this is consistent with the result of \citet{Yamazaki2015a}. 
    In their work, they found a similar relationship between the cusp density enhancement 
    and IMF $B_y$, although they argued that other processes are required to fully interpret 
    the phenomenon.

    
\bibliography{/home/guod/work/bib/literature}
\bibliographystyle{/home/guod/Documents/template/agu_template/BibTeX/agufull08}
\end{article}
%------------------------------
\begin{figure}
    \centering
    \noindent\includegraphics[width=0.8\textwidth]{/home/guod/Documents/work/work2paper/figures/fig1.eps}
    \caption{Superposed epoch results of $Bx$, $By$ and $Bz$ components of IMF in GSM coordinate system and the
    geomagnetic auroral electrojet index $AE$ as a function of month and epoch time during 2002-2010. The epoch 
    time centers at 0000 UT on the reversing day when the solar wind sector polarity changes from away from to 
    toward (Away-Toward) or from toward to away from (Toward-Away) the Sun. }
    \label{figure1}
\end{figure}
\begin{figure}
    \centering
    \noindent\includegraphics[width=0.8\textwidth]{/home/guod/Documents/work/work2paper/figures/fig2.eps}
    \caption{Relative changes in thermospheric densities near the geographic north (N) and south (S) poles 
    ($\left|latitude\right| > 88.5^\circ$) versus epoch time. The thermospheric density data during 2002-2010 
    are sorted in month and solar wind sector polarity. The blue lines represent the median values, and the 
    gray dashed lines represent the upper and lower quartiles.}
    \label{figure2}
\end{figure}
%\begin{figure}
%    \centering
%    \noindent\includegraphics[width=0.8\textwidth]{/fig26.eps}
%    \caption{ Magnetic local time changes versus universal time for the north and south geographical poles, 
%    respectively. The blue vertical line in the right panel at 1600 UT represents the universal time when 
%    the density near the south pole maximizes as shown in Figure 2.}
%    \label{figure3}
%\end{figure}
\begin{figure}
    \centering
    \noindent\includegraphics[width=0.8\textwidth]{/home/guod/Documents/work/work2paper/figures/fig3.eps}
    \caption{The same as Figure 2, but the time intervals are constrained to years during 2002-2005 
    and 2006-2010 for the upper and lower panels, respectively}
    \label{figure3}
\end{figure}
%------------------------------
%...as shown by \textit{Kilby} [2008].
%...as shown by {\textit  {Lewin}} [1976], {\textit  {Carson}} [1986], {\textit  {Bartholdy and Billi}} [2002], and {\textit  {Rinaldi}} [2003].
%...has been shown [\textit{Kilby et al.}, 2008].
%...has been shown [{\textit  {Lewin}}, 1976; {\textit  {Carson}}, 1986; {\textit  {Bartholdy and Billi}}, 2002; {\textit  {Rinaldi}}, 2003].
%...has been shown [e.g., {\textit  {Lewin}}, 1976; {\textit  {Carson}}, 1986; {\textit  {Bartholdy and Billi}}, 2002; {\textit  {Rinaldi}}, 2003].

%...as shown by \citet{jskilby}.
%...as shown by \citet{lewin76}, \citet{carson86}, \citet{bartoldy02}, and \citet{rinaldi03}.
%...has been shown \citep{jskilbye}.
%...has been shown \citep{lewin76,carson86,bartoldy02,rinaldi03}.
%...has been shown \citep [e.g.,][]{lewin76,carson86,bartoldy02,rinaldi03}.
% Please use ONLY \citet and \citep for reference citations.
%------------------------------
\end{document}
