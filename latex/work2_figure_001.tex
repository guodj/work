\documentclass[12pt,a4paper]{article}
\usepackage{geometry}    %  页面控制
\geometry{left=3cm, right=3cm, top=1.0cm, bottom=3cm}
\usepackage{graphicx}  %插入图片
\DeclareGraphicsExtensions{.png,.eps}

\begin{document}

\section{Physical picture}
\begin{figure}[!ht]
    \centering
    \includegraphics[width=0.8\textwidth]{/data/output/fig/work2/fig1}
    \label{fig1}
    \caption{CHAMP and GRACE LT in every day. For some days, the results are wrong. From f12() in $\sim$/work/python/work2.py}
\end{figure}

\begin{figure}[!ht]
    \centering
    \includegraphics[width=0.8\textwidth]{/data/output/fig/work2/fig7}
    \label{fig2}
    \caption{ CHAMP MLT/MLat and LT/Lat distributions in January 1-10, 2005. Note the difference of Mlat between the Northern and Southern hemisphere. From f17() in $\sim$/work/python/work2.py}
\end{figure}

\begin{figure}[!ht]
    \centering
    \includegraphics[width=0.8\textwidth]{/data/output/fig/work2/fig4}
    \label{fig3}
    \caption{Test griddata and contourf. For griddata, both the bin size and the units of x,y are important. For contourf, the units of x,y have no effect. From f11() in $\sim$/work/python/work2.py}
\end{figure}

\begin{figure}[!ht]
    \centering
    \includegraphics[width=0.8\textwidth]{/data/output/fig/work2/fig6}
    \label{fig4}
    \caption{Schematic of Mlat/Mlon(Mlt) in apex coordinates, and Lat/lon in geodetic coordinates. from f3() in $\sim$/work/python/sth.py}
\end{figure}
\newpage

\section{work2}
Superposed epoch analysis of the thermosphere density response to sector boundary crossings.
How to use the magnetic apex coordinate?
\begin{figure}[!ht]
    \centering
    \includegraphics[width=0.8\textwidth]{/data/output/fig/work2/fig2}
    \label{fig5}
    \caption{Imf and ap changes as a function of month and epoch time centered at sector boundary crossing date, from f15() in $\sim$/work/python/work2.py}
\end{figure}

\begin{figure}[!ht]
    \centering
    \includegraphics[width=0.8\textwidth]{/data/output/fig/work2/fig3}
    \label{fig6}
    \caption{CHAMP and GRACE density variations versus day of year and epoch days centered at sector boundary crossing date. The CIR and CME effects are not excluded. Away-toward: 520 up or down cases; toward-away: 534 up or down cases. From f16() in $\sim$/work/python/work2.py. }
\end{figure}

\begin{figure}[!ht]
    \centering
    \includegraphics[width=0.8\textwidth]{/data/output/fig/work2/fig12}
    \label{fig7}
    \caption{From top to bottom are for Northern high latitudes, low-middle latitudes and Southern high latitudes. For away-toward condition, Feb-Apr data are selected; for toward-away condition, Aug-Oct data are selected. From f16() in $\sim$/work/python/work2.py.}
\end{figure}

\begin{figure}[!ht]
    \centering
    \includegraphics[width=0.8\textwidth]{/data/output/fig/work2/fig18}
    \label{fig8}
    \caption{Density variations in the Southern high latitudes (Aug-Oct) during +/- 5 days of toward-away sectors. From f18() in $\sim$/work/python/work2.py.}
\end{figure}

\begin{figure}[!ht]
    \centering
    \includegraphics[width=0.8\textwidth]{/data/output/fig/work2/fig24}
    \label{fig9}
    \caption{Density variations in the Southern high latitudes (Aug-Oct) during +/- 5 days of toward-away sectors. From f18() in $\sim$/work/python/work2.py.}
\end{figure}

\begin{figure}[!ht]
    \centering
    \includegraphics[width=0.8\textwidth]{/data/output/fig/work2/fig25}
    \caption{Grace density changes at geographic poles. From f24() in $\sim$/work/python/work2.py.}
    \label{fig10}
\end{figure}

\begin{figure}[!ht]
    \centering
    \includegraphics[width=0.8\textwidth]{/data/output/fig/work2/fig26}
    \label{fig11}
    \caption{Grace MLT changes at geographic poles. From f24() in $\sim$/work/python/work2.py.}
\end{figure}

\begin{figure}[!ht]
    \centering
    \includegraphics[width=0.8\textwidth]{/data/output/fig/work2/fig32}
    \label{fig12}
    \caption{Grace density changes at geographic poles. Both results for the solar maximum and minimum are shown. From f24() in $\sim$/work/python/work2.py.}
\end{figure}

\end{document}
