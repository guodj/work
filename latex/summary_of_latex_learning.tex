\documentclass[11pt,a4paper]{article}
% pt is a unit to measure font size in typographic system
\usepackage[dvips]{graphicx}
\begin{document}
%\listoftables
\title{Summary of \LaTeX{} Learning}
\author{Dongjie Guo}
\date{\today}
\maketitle
\section{Things You Need to Know}
\subsection{The Name of the Game}
\TeX{} is a computer program aimed at typesetting text and mathematical formulae, 
which is created by Donald E. Knuth. \LaTeX{} is a macro package which was originally
written by Leslie Lamport. It uses the \TeX{} formatter as its typesetting engine.
\subsection{\LaTeX{} Input Files}
\subsubsection{Spaces}
Several consecutive whitespace characters are treated as one ``space''. Whitespace
at the start of a line is generally ignored, and a single linebreak is treated as 
``whitespace''. An empty line starts a new paragraph.
\subsubsection{Special Characters}
\$ \& \% \# \_ \{ \} \~{} \^{} $\backslash$\\
$\backslash$$\backslash$ is used for linebreaking.\newline
$\backslash$newline starts a new line.
\subsubsection{\LaTeX{} Commands}
\LaTeX{} commands have two formats:\\
\begin{itemize}
    \item$\backslash$$+$ a few letters. Command names are terminated by a space,
        a number or any other `non-letter'. \LaTeX{} ignores whitespace after the 
        commands. Put \{\} and a blank after a command if you want to get a space
        after the command.
    \item$\backslash$$+$ one special character.\\
\end{itemize}
The parameter of commands which has to be given is in the curly braces \{\},
optional parameters are added in the square brackets [].\\
\subsubsection{Comments}
When \LaTeX{} encounters a \% character, 
it ignores the rest of the present line, the linebreak,
and all whitespace at the beginning of the next line.
\subsection{Input File Structure}
$\backslash$documentclass\{\ldots\}\\
The area between $\backslash$documentclass and $\backslash$begin\{document\} is called preamble.
$\backslash$usepackage\{\ldots\}\\
$\backslash$begin\{document\}\\
$\backslash$end\{document\}
\section{The Layout of th Document}
\subsection{Document Classes}
$\backslash$documentclass[\textit{options}]\{\textit{class}\}\\
Here \textit{class} can be article, report, book, and slides. \textit{Options} 
customises the behaviour of the document class.
\subsection{Page Style}
\LaTeX{} supports 3 predefined header/footer combinations.\\
$\backslash$pagestyle\{\textsl{style}\}, where
\textsl{style} can be \textsl{plain}, \textsl{headings}, \textsl{empty}.\\
$\backslash$thispagestyle\{\textsl{style}\} changes the current page style.
\section{Typesetting Text}
\subsection{The Structure of Text and Language}
The most important text unit in \LaTeX{} is the paragraph which should reflect
one coherent idea.

When you read the sentence aloud, a comma should be inserted where you feel 
the urge to breathe.
\subsection{Linebreaking and Pagebreaking}
\subsubsection{Justified Paragraphs}
$\backslash$$\backslash$ or $\backslash$newline\\ starts a new line 
without starting a new paragraph.\\
$\backslash$$\backslash$* \\additionally prohibits a pagebreak after the forced
linebreak.\\
$\backslash$newpage \\starts a new page.\\

What does$\backslash$linebreak[3] mean?\linebreak[3]
\subsubsection{Hyphenation}
The command $\backslash$ hyphenation\{word list\} causes the words listed in the
argument to be hyphenated only at the points marked by "-".
$\backslash$- inserts a discretionary hyphen into a word. This also becomes the only
point hyphenation is allowed in this word.\\

Example:
\begin{quote}
    $\backslash$hyphenation\{FORTRAN Hy-phen-a-tion\}.
\end{quote}
This example above will allow ``hyphenation'' to be hyphenated as well as 
``Hyphenation'', and it prevents ``FORTRAN'', ``Fortran'', and ``fortran'' from
being hyphenated at all.

$\backslash$mbox\{text\} keeps several words together on one line.
\subsection{Ready Made Strings}
$\backslash$today, $\backslash$Tex, $\backslash$LaTeX, $\backslash$LaTeXe
\subsection{Special Characters and Symbols}
\subsubsection{Quotation Marks}
use two `s for opening quotation marks and two 's for closing quotation marks
''I am a quotation!''
\subsubsection{Dashes and Hyphens}
4 kinds of dashes. 3 of them can be accessed with different numbers of consecutive 
dashes. The fourth is the mathematical minus sign: daughter-in-law,pages 13--67,
yes---or no?, $-1$
\subsubsection{Tilde($\sim$)}
In the web addresses, use \$$\backslash$sim\$ to stand for $\sim$.
\subsubsection{Ellipsis($\ldots$)}
Use $\backslash$ldots for \ldots; Beijing, New York, Tokyo, \ldots
\subsubsection{Ligatures}
The so-called ligatures can be prohibited by inserting $\backslash$mbox\{\} between 
the two letters in question. Not shelfful, but shelf\mbox{}ful
\subsubsection{Accents and Special Characters}
See below:\\
H\^otel, na\"\i ve, \` el\'eve,
sm\o rrebr\o d, !`Se\~norita!,
sch\"onbrunner Schlo\ss{} stra\ss e
\subsection{The Space between Words}
Default: \LaTeX{} inserts slightly more space at the end of a sentence. If a period 
follows an uppercase letter, this is not taken as a sentence ending.

Exception has to be specified. A backslash in front of a space which will not be 
enlarged. A tilde `\~{}' character generates a space which cannot be enlarged and 
which additionally prohibits a linebreak. The command $\backslash$@ in front of a 
period specifies that this period terminates a sentence even when it follows an 
uppercase letter.

$\backslash$frenchspacing tells \LaTeX{} not to insert more space after a period than
after ordinary charater.
\subsection{Section}
Available sectioning commands for the article class:\\
\begin{quote}
    $\backslash$section\{\ldots\}\\
    $\backslash$subsection\{\ldots\}\\
    $\backslash$subsubsection\{\ldots\}\\
    $\backslash$paragraph\{\ldots\}\\
    $\backslash$subparagraph\{\ldots\}\\
    $\backslash$appendix\\
\end{quote}
For the report and book class,
$\backslash$part\{\ldots\} and $\backslash$chapter\{\ldots\} are available.

$\backslash$tableofcontents expands to a table of contents at the place where it is 
issued. Sometimes it might be necessary to compile the document a third time to get a 
correct table of contents. ``Starred'' versions of sectioning commands such as 
$\backslash$section*\{Help\} do not show up in the table of contents and is 
not numbered. 
\subsection{Title}
The title of the whole document is generated by issuing a\\ 
$\backslash$maketitle \\
command,and its contents have to be defined by commands\\
$\backslash$title\{\ldots\}, $\backslash$author\{\ldots\} and optionally 
$\backslash$date\{\ldots\} \\
before calling $\backslash$maketitle. Supply sereral names seperated by 
$\backslash$and commands.
\subsection{Cross References}\label{CR}
\LaTeX{} provides the following commands for cross referencing:\\
$\backslash$label\{\textsl{marker}\}, $\backslash$ref\{\textsl{marker}\}, 
$\backslash$ref\{\textsl{marker}\}.\\
Example: ``see section~\ref{CR} on page~\pageref{CR}.''\\
In fact, $\backslash$label just saves the last automatically generated number.
\subsection{Footnotes}
With the command $\backslash$footnote\{\textit{footnote text}\}, a footnote is 
printed at the foot of the current page.
Example: Footnotes\footnote{This is a footnote.} are often used.
\subsection{Emphasized Words}
\emph{This is \emph{Emphasized}}. Words are emphasized by typesetting them in an
\textit{italic} font. You can also specify the emphasized words \textsf{in a 
\emph{sans-serif} font,} \texttt{ or in \emph{typewriter} style. }
\subsection{Environments}
\subsubsection{Itemize,Enumerate, and Description}
What is the difference between them?

This is $\backslash$itemize:\\
\begin{itemize}
    \item This is itemize
    \item This is itemize    
\end{itemize}
This is $\backslash$enumerate:\\
\begin{enumerate}
    \item This is enumerate
    \item This is enumerate
\end{enumerate}
This is $\backslash$description:
\begin{description}
    \item This is description
    \item This is description
\end{description}
\subsubsection{Flushleft, Flushright, and Center}
This is $\backslash$flushleft:
\begin{flushleft}
    This text is \\left-aligned. \LaTeX{} is not trying to make each line the same
    length.
\end{flushleft}
\subsubsection{Quote, Quotation, and Verse}
A quote is like this:
\begin{flushleft}
    \begin{tabular}{|p{6cm}|}
        \hline
        A typographical rule of thumb for the line length is:
        \begin{quote}
            No line should contain more than 66~characters.

            This is why \LaTeX{} pages have such large borders by default.
        \end{quote}
        That's why multicolumn print is often used in newspapers.\\
        \hline
    \end{tabular}\\
\end{flushleft}
The quote does not indent paragraphs, while the quotation does.

A verse is like this:
\begin{flushleft}
    \begin{tabular}{|p{6cm}|}
        \hline
        I know only one English poem by heart. It is about Humpty Dumpty.
        \begin{verse}
            Humpty Dumpty sat on a wall:\\
            Humpty Dumpty had a great fall.\\
            All the King's horses and all the King's men\\
            Couldn't put Humpty together again.
        \end{verse}\\
        \hline
    \end{tabular}\\
\end{flushleft}
\subsubsection{Tabular}
The \textit{table spec} argument of the
\begin{flushleft}
    \begin{tabular}{|c|}
        \hline
        $\backslash$begin\{tabular\}\{\textit{table spec}\}\\
        \hline
    \end{tabular}\\
\end{flushleft}
command defines the format of the table. Use l for a column of left-aligned text,
r for right-aligned text, and c for centered test; p\{\textit{width}\} for a
column containing justified text with linebreaks, and $|$ for a vertical line.

Within a tabular environment, \& jumps to the next column, $\backslash$$\backslash$
starts a new line and $\backslash$hline inserts a horizontal line.\\
\begin{center}
    \begin{tabular}{@{} l @{...} r @{}}
        \hline
        gdj & lj \\
        lj & gdj \\
        l & i \\
        \hline
    \end{tabular}
\end{center}

Since there is no build-in way to align numeric columns to a decimal point, we
can ``cheat'' and do it by using two columns: a right-aligned integer and a left-
aligned fraction. 

A column label can be placed above our numeric ``columns'' by using 
the $\backslash$multicolumn commands.\\
\begin{center}
    \begin{tabular}{|c|c|}
        \hline
        \multicolumn{2}{|c|}{\textbf{Ene}}\\
        \hline
        Mene & Muh! \\
        \hline
    \end{tabular}
\end{center}
\subsection{Floating Bodies}
There are two environments for floating bodies; one for tables and one for figures.
\begin{table}[!h]
    \begin{tabular}{c}
        \hline
$\backslash$begin\{figure\}[\textit{placement specifier}] or
$\backslash$begin\{table\}[\textit{placement specifier}]\\
\hline
\end{tabular}
\end{table}\\
\textit{Placement specifier} can be h, t, b, p, !. See table \ref{t1} for detail.
\begin{table}[!h]
    \caption{Float Placing Permissions.}\label{t1}
    \begin{tabular}{c  p{10cm}}
        Spec & Permission to place the float\ldots\\
        \hline
        h    & \emph{here} at the very place in the text where it occurred. This is
        useful mainly for small floats.\\
        t    & at the \emph{top} of a page\\
        b    & at the \emph{bottom} of a page\\
        p    & on a special \emph{page} containing only floats.\\
        !    & without considering most of the internal parameters which could
        stop this float from being placed.\\
        \hline
    \end{tabular}
\end{table}

The two commands
\begin{flushleft}
    $\backslash$listoffigures and $\backslash$listoftables
\end{flushleft}
operate analogously to the $\backslash$tableofcontents command, printing a list of 
figures or tables, respectively.
\section{Specialities}
\subsection{Including EPS Graphics}
The step to include a picture into your document:\\
\begin{enumerate}
    \item Load the \textbf{graphicx} package in the preamble with\\
        $\backslash$usepackage[\textit{driver}]\{graphicx\}\\
        where \textit{driver} is the name of ``dvi to postscript'' converter program.
        It helps the \textbf{graphicx} package choose the correct method to insert 
        information about the graphics into the .dvi file, so that the printer
        understands it and can correctly include the .eps file.
    \item Use the command\\
        $\backslash$includegraphics[\textit{key=value},\ldots]\{\textit{file}\}\\
        Table \ref{t2} lists the most important keys.\\
        \begin{table}[!h]
            \begin{center}
                \caption{Key Names for \textsf{graphicx} Package}\label{t2}
                \begin{tabular}{l l}
                    \hline
                    width  & scale graphic to the specified width\\
                    height & scale graphic to the specified height\\
                    angle  & rotate graphic counterclockwise\\
                    scale  & scale graphic\\
                    \hline
                \end{tabular}
            \end{center}
        \end{table}
\end{enumerate}
\section{Bibliography}
The structure of Bibliography:\\
\begin{description}
    \item$\backslash$begin\{thebibliography\}\{99\}\\
    \item$\backslash$bibitem\{pa\} H.~Partl:\\
    \item$\backslash$emph\{German $\backslash$TeX\},
        TuGboat Vol.\~{}9, No.\~{}1 ('88)\\
    \item$\backslash$end\{thebibliography\}\\
\end{description}
Then you can cite the paper by: Partl\~{}$\backslash$cite\{pa\} has\ldots
\end{document}
